% Created 2022-08-18 qui 09:30
% Intended LaTeX compiler: pdflatex
\documentclass[t, aspectratio=169]{beamer}
\usepackage[utf8]{inputenc}
\usepackage[T1]{fontenc}
\usepackage{graphicx}
\usepackage{longtable}
\usepackage{wrapfig}
\usepackage{rotating}
\usepackage[normalem]{ulem}
\usepackage{amsmath}
\usepackage{amssymb}
\usepackage{capt-of}
\usepackage{hyperref}
\usepackage[newfloat]{minted}
\usepackage{tikz}
\usepackage{booktabs}
\usetheme{default}
\author{Luigi D. C. Soares}
\date{DCC/UFMG (25/08/2022)}
\title{Variáveis e Tipos}
\subtitle{Progamação e Desenvolvimento de Software I}
\title[Variáveis e Tipos]{Variáveis e Tipos}
\subtitle{Programação e Desenvolvimento de Software I}
\author[\tiny\{gleison.mendonca, luigi.domenico\}@dcc.ufmg.br]{%
Gleison S. D. Mendonça, Luigi D. C. Soares\texorpdfstring{\\}{}
\texttt{\{gleison.mendonca, luigi.domenico\}@dcc.ufmg.br}}
\institute[DCC/UFMG]{}
\date[25/08/2022]{}
%\usetheme{saori}
%\usemintedstyle{native}
\usetheme{ufmg}
\hypersetup{
 pdfauthor={Luigi D. C. Soares},
 pdftitle={Variáveis e Tipos},
 pdfkeywords={},
 pdfsubject={},
 pdfcreator={Emacs 28.1 (Org mode 9.6)}, 
 pdflang={English}}
\begin{document}

\maketitle

\begin{frame}[label={sec:org8463d7a}]{Variáveis}
\begin{itemize}
\item Na matemática
\begin{itemize}
\item É uma entidade capaz de representar um valor ou expressão
\item Pode representar um número ou um conjunto de números
\item \(f(x) = x²\)
\end{itemize}
\end{itemize}
\end{frame}

\begin{frame}[label={sec:org4ab6ca1}]{Variáveis}
\begin{itemize}
\item Na matemática
\begin{itemize}
\item É uma entidade capaz de representar um valor ou expressão
\item Pode representar um número ou um conjunto de números
\item \(f(x) = x²\)
\end{itemize}

\item Na computação
\begin{itemize}
\item Posição de memória que armazena uma informação
\item Pode ser modificada pelo programa
\end{itemize}
\end{itemize}
\end{frame}

\begin{frame}[label={sec:org8298b02},fragile]{Memória}
 \begin{itemize}
\item \texttt{int x = 10;}
\item \texttt{double y = 2.5;}
\end{itemize}

\begin{center}
\begin{tabular}{ccc}
\toprule
Nome & \alert{Valor} & \alert{Endereço}\\
\midrule
 &  & 0\\
x & 10 & 4\\
y & 2.5 & 8\\
 &  & 12\\
 &  & 16\\
\bottomrule
\end{tabular}
\end{center}
\end{frame}
\end{document}
