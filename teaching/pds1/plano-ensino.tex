% Created 2022-08-19 sex 16:24
% Intended LaTeX compiler: pdflatex
\documentclass[a4paper, 11pt]{article}
\usepackage[utf8]{inputenc}
\usepackage[T1]{fontenc}
\usepackage{graphicx}
\usepackage{longtable}
\usepackage{wrapfig}
\usepackage{rotating}
\usepackage[normalem]{ulem}
\usepackage{amsmath}
\usepackage{amssymb}
\usepackage{capt-of}
\usepackage{hyperref}
\usepackage[newfloat]{minted}
\usepackage[brazil, portuges]{babel}
\usepackage[utf8]{inputenc}
\usepackage{fancyhdr}
\usepackage[margin=1.2in]{geometry}
\usepackage[table]{xcolor}
\usepackage{booktabs}
\usepackage{array}
\usepackage{enumitem}
\usepackage{xcolor}
\usepackage{datetime2}
\makeatletter
\DeclareRobustCommand*\course[1]{\gdef\@course{#1}}
\DeclareRobustCommand*\institution[1]{\gdef\@institution{#1}}
\DeclareRobustCommand*\semester[1]{\gdef\@semester{#1}}
\title{Plano de Ensino / Cronograma}
\author{Profs. Gleison S. D. Mendonça e Luigi D. C. Soares}
\course{Programação e Desenvolvimento de Software I}
\institution{DCC / ICEx / UFMG}
\semester{2022.2}
\let\thetitle\@title{}
\let\theauthor\@author{}
\let\thecourse\@course{}
\let\theinstitution\@institution{}
\let\thesemester\@semester{}
\let\thedate\@date{}
\makeatother
\DTMnewdatestyle{brDateStyle}{%
\renewcommand{\DTMdisplaydate}[4]{##3/##2/##1}%
\renewcommand{\DTMDisplaydate}{\DTMdisplaydate}}
\DTMsetdatestyle{brDateStyle}
\pagestyle{fancy}
\fancyhf{}
\setlength{\headheight}{15pt}
\lhead{\theauthor \\ \thecourse}
\rhead{\theinstitution \\ \thesemester}
\rfoot{\thepage}
\hypersetup{
colorlinks,
linkcolor={red!50!black},
citecolor={blue!50!black},
urlcolor={blue!80!black}
}
\date{\today}
\title{}
\hypersetup{
 pdfauthor={},
 pdftitle={},
 pdfkeywords={},
 pdfsubject={},
 pdfcreator={Emacs 28.1 (Org mode 9.6)}, 
 pdflang={Portuges}}
\begin{document}

\begin{center}
\Large\bfseries\thetitle
\end{center}

\section{Informações Gerais}
\label{sec:org9068452}

\setlist{leftmargin=1.5em, itemsep=0em}
\begin{description}
\item[{Código da disciplina:}] DCC203
\item[{Semestre:}] 2022.2
\item[{Professores:}] Luigi Domenico Cecchini Soares e Gleison Souza Diniz Mendonça
\item[{Contatos:}] (Adicionar [PDS 1] no assunto)
\begin{itemize}
\item Turmas \textbf{TA1} e \textbf{TF}:  \href{mailto:gleison.mendonca@dcc.ufmg.br}{gleison.mendonca@dcc.ufmg.br}
\item Turmas \textbf{TA2} e \textbf{TM3}:  \href{mailto:luigi.domenico@dcc.ufmg.br}{luigi.domenico@dcc.ufmg.br}
\end{itemize}
\item[{Horários:}] \phantom{}
\begin{itemize}
\item Turma \textbf{TA1} e \textbf{TA2}: 3a e 5a, 07:30 -- 09:10
\item Turma \textbf{TM3}: 3a e 5a, 13:00 -- 14:40
\item Turma \textbf{TF}: 3a 20:55 -- 22:35, 5a 19:00 -- 20:40
\end{itemize}
\item[{Salas:}] (a definir)
\end{description}

\section{Ementa}
\label{sec:org47ad5ad}
Introdução ao funcionamento de um computador e ao desenvolvimento de
programas. Desenvolvimento de programas em uma linguagem de alto nível.  Tipos
de dados simples, apontadores, variáveis compostas homogêneas e heterogêneas.
Entrada e saída. Estruturas de controle e repetição. Funções e ferramentas de
modularização.

\section{Objetivos}
\label{sec:orgf8af3e8}
Os principais objetivos da disciplina são possibilitar ao aluno:
\begin{itemize}
\item O conhecimento dos princípios de estrutura e funcionamento do computador;
\item O domínio de técnicas de resolução de problemas por computador (técnicas de
desenvolvimento de algoritmos), através de aulas teóricas e aulas em
laboratório;
\item A utilização do computador para a resolução de problemas computacionais
através da implementação de algoritmos.
\end{itemize}

\section{Avaliação}
\label{sec:org488da6c}

\begin{itemize}
\item Provas (3 x 25 pontos): 75 pontos
\item Atividades práticas (VPLs): 10 pontos
\item Trabalhos práticos: 15 pontos
\end{itemize}

\section{Bibliografia}
\label{sec:org9c00618}

\begin{itemize}
\item Livro-texto: Linguagem C completa e descomplicada, André Backes \textbf{[A]}
\item Complementar:
\vspace{-0.5em}
\begin{itemize}
\item Introdução às Estruturas de Dados, Waldemar Celes
\item Projeto de Algoritmos com implementação em Pascal e C, 3a edição,
Nivio Ziviani
\item Algoritmos estruturados, 3a edição, Harry Farrer, Becker, Faria, Matos,
dos Santos, Maia
\end{itemize}
\end{itemize}

\section{Cronograma (\color{red}\bfseries Última atualização: \today)}
\label{sec:org1498586}

\fcolorbox{black}{green!25}{\rule{0pt}{6pt}\rule{6pt}{0pt}}\quad Não há aula \qquad
\fcolorbox{black}{gray!25}{\rule{0pt}{6pt}\rule{6pt}{0pt}}\quad Sala de Aula \qquad
\fcolorbox{black}{yellow!25}{\rule{0pt}{6pt}\rule{6pt}{0pt}}\quad Laboratório \qquad
\fcolorbox{black}{red!15}{\rule{0pt}{6pt}\rule{6pt}{0pt}}\quad Prova

\begin{longtable}{>{\bfseries}ccl>{\bfseries}cc}
\toprule
\textbf{Aula} & \textbf{Data} & \textbf{Tópico} & \textbf{Trabalhos Práticos} & \textbf{Ref.}\\
\midrule
\endfirsthead
\multicolumn{5}{l}{Continuação da página anterior} \\
\toprule

\textbf{Aula} & \textbf{Data} & \textbf{Tópico} & \textbf{Trabalhos Práticos} & \textbf{Ref.} \\

\midrule
\endhead
\midrule\multicolumn{5}{r}{Continua na página seguinte} \\
\endfoot
\endlastfoot
\rowcolor{green!25} & 23/08 & Não haverá aula: Recepção calouros &  & \\
\rowcolor{yellow!25} 01 & 25/08 & Apresentação do curso + Introdução &  & \\
\rowcolor{gray!25} 02 & 30/08 & Variáveis, Tipos e Entrada/Saída &  & A.2\\
\rowcolor{yellow!25} 03 & 01/09 & Prática 01 &  & \\
\rowcolor{gray!25} 04 & 06/09 & Operadores &  & A.3\\
\rowcolor{yellow!25} 05 & 08/09 & Prática 02 &  & \\
\rowcolor{gray!25} 06 & 13/09 & Comandos condicionais &  & A.4\\
\rowcolor{yellow!25} 07 & 15/09 & Prática 03 &  & \\
\rowcolor{gray!25} 08 & 20/09 & Comando de repetição + Prática 04 &  & A.5\\
\rowcolor{yellow!25} 09 & 22/09 & Revisão / Dúvidas &  & \\
\rowcolor{red!15} 10 & 27/09 & \textbf{Prova teórica 01} &  & \\
\rowcolor{yellow!25} 11 & 29/09 & Correção Prova 01 & Divulgação TP 01 & \\
\rowcolor{gray!25} 12 & 04/10 & Arranjos e Strings &  & A.6, A.7\\
\rowcolor{yellow!25} 13 & 06/10 & Prática 05 &  & \\
\rowcolor{gray!25} 14 & 11/10 & Tipos definidos pelo programador &  & A.8\\
\rowcolor{yellow!25} 15 & 13/10 & Prática 06 &  & \\
\rowcolor{gray!25} 16 & 18/10 & Módulos e Funções &  & A.9\\
\rowcolor{yellow!25} 17 & 20/10 & Prática 07 &  & \\
\rowcolor{gray!25} 18 & 25/10 & Ponteiros + Prática 08 & Entrega TP 01 & A.9, A.10\\
\rowcolor{yellow!25} 19 & 27/10 & Revisão / Dúvidas &  & \\
\rowcolor{red!15} 20 & 01/11 & \textbf{Prova teórica 02} &  & \\
\rowcolor{yellow!25} 21 & 03/11 & Correção Prova 02 & Divulgação TP 02 & \\
\rowcolor{gray!25} 22 & 08/11 & Alocação Dinâmica &  & A.11\\
\rowcolor{yellow!25} 23 & 10/11 & Prática 09 &  & \\
\rowcolor{green!25} & 15/11 & Feriado: Proclamação da República &  & \\
\rowcolor{yellow!25} 24 & 17/11 & Prática 10 &  & \\
\rowcolor{gray!25} 25 & 22/11 & Recursão &  & A.9\\
\rowcolor{yellow!25} 26 & 24/11 & Prática 11 &  & \\
\rowcolor{gray!25} 27 & 29/11 & Arquivos + Prática 12 & Entrega TP 02 & A.12\\
\rowcolor{yellow!25} 28 & 01/12 & Revisão / Dúvidas &  & \\
\rowcolor{red!15} 29 & 06/12 & \textbf{Prova teórica 03} &  & \\
\rowcolor{green!25} & 08/12 & Feriado: Nossa Senhora da Conceição &  & \\
\rowcolor{green!25} & 13/12 & Não haverá aula: Semifinal da Copa &  & \\
\rowcolor{red!15} 30 & 15/12 & \textbf{Prova substitutiva} &  & \\
 &  &  &  & \\
\rowcolor{red!15} &  & \textbf{Exame especial} &  & \\
\bottomrule
\end{longtable}
\end{document}
