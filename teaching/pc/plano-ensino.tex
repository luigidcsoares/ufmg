% Created 2022-08-19 sex 14:23
% Intended LaTeX compiler: pdflatex
\documentclass[a4paper, 11pt]{article}
\usepackage[utf8]{inputenc}
\usepackage[T1]{fontenc}
\usepackage{graphicx}
\usepackage{longtable}
\usepackage{wrapfig}
\usepackage{rotating}
\usepackage[normalem]{ulem}
\usepackage{amsmath}
\usepackage{amssymb}
\usepackage{capt-of}
\usepackage{hyperref}
\usepackage[newfloat]{minted}
\usepackage[brazil, portuges]{babel}
\usepackage[utf8]{inputenc}
\usepackage{fancyhdr}
\usepackage[margin=1.2in]{geometry}
\usepackage[table]{xcolor}
\usepackage{booktabs}
\usepackage{array}
\usepackage{enumitem}
\usepackage{xcolor}
\usepackage{datetime2}
\makeatletter
\DeclareRobustCommand*\course[1]{\gdef\@course{#1}}
\DeclareRobustCommand*\institution[1]{\gdef\@institution{#1}}
\DeclareRobustCommand*\semester[1]{\gdef\@semester{#1}}
\title{Plano de Ensino / Cronograma}
\author{Prof. Luigi Domenico Cecchini Soares}
\course{Programação de Computadores}
\institution{DCC / ICEx / UFMG}
\semester{2022.2}
\let\thetitle\@title{}
\let\theauthor\@author{}
\let\thecourse\@course{}
\let\theinstitution\@institution{}
\let\thesemester\@semester{}
\let\thedate\@date{}
\makeatother
\DTMnewdatestyle{brDateStyle}{%
\renewcommand{\DTMdisplaydate}[4]{##3/##2/##1}%
\renewcommand{\DTMDisplaydate}{\DTMdisplaydate}}
\DTMsetdatestyle{brDateStyle}
\pagestyle{fancy}
\fancyhf{}
\setlength{\headheight}{15pt}
\lhead{\theauthor \\ \thecourse}
\rhead{\theinstitution \\ \thesemester}
\rfoot{\thepage}
\hypersetup{
colorlinks,
linkcolor={red!50!black},
citecolor={blue!50!black},
urlcolor={blue!80!black}
}
\date{\today}
\title{}
\hypersetup{
 pdfauthor={},
 pdftitle={},
 pdfkeywords={},
 pdfsubject={},
 pdfcreator={Emacs 28.1 (Org mode 9.6)}, 
 pdflang={Portuges}}
\begin{document}

\begin{center}
\Large\bfseries\thetitle
\end{center}

\section{Informações Gerais}
\label{sec:orgc2a6a45}

\setlist{leftmargin=1.5em, itemsep=0em}
\begin{description}
\item[{Código da disciplina:}] DCC001 / DCC208
\item[{Semestre:}] 2022.2
\item[{Professor:}] Luigi Domenico Cecchini Soares
\item[{Contato:}] \href{mailto:luigi.domenico@dcc.ufmg.br}{luigi.domenico@dcc.ufmg.br} (Adicionar [PC] no assunto)
\item[{Horários:}] 3a 19:00 -- 20:40, 5a 20:55 -- 22:35
\item[{Salas:}] (a definir)
\end{description}

\section{Ementa}
\label{sec:orgffb470a}
Metodologia de desenvolvimento de programas. Programação em Linguagem de
Alto Nível. Comandos Básicos. Estruturas de dados. Modularização. Bibliotecas
científicas.

\section{Avaliação}
\label{sec:org3e387bc}

\begin{itemize}
\item Avaliações (4 x 25 pontos): 75 pontos (serão consideradas as 3 maiores notas)
\item Atividades práticas (VPLs): 15 pontos
\item Projeto final: 10 pontos
\end{itemize}

\section{Bibliografia}
\label{sec:orgfac2ac2}

\begin{itemize}
\item MENEZES, Nilo Ney Coutinho. Introdução à programação com Python–2ª edição:
Algoritmos e lógica de programação para iniciantes. Novatec Editora, 2016
\item SEVERANCE, Charles. Python for informatics: Exploring information .
CreateSpace, 2013. (disponível gratuitamente em \url{http://www.pythonlearn.com/book.php})
\item MCKINNEY, Wes. Python for data analysis: Data wrangling with Pandas, NumPy,
and IPython. " O'Reilly Media, Inc.", 2012.
\end{itemize}

\section{Cronograma (\color{red}\bfseries Última atualização: \today)}
\label{sec:org0dbfbdd}

\fcolorbox{black}{green!25}{\rule{0pt}{6pt}\rule{6pt}{0pt}}\quad Não há aula \qquad
\fcolorbox{black}{gray!25}{\rule{0pt}{6pt}\rule{6pt}{0pt}}\quad Sala de Aula \qquad
\fcolorbox{black}{yellow!25}{\rule{0pt}{6pt}\rule{6pt}{0pt}}\quad Laboratório \qquad
\fcolorbox{black}{red!15}{\rule{0pt}{6pt}\rule{6pt}{0pt}}\quad Avaliação

\begin{longtable}{>{\bfseries}ccp{6cm}cc}
\toprule
\textbf{Aula} & \textbf{Data} & \textbf{Conteúdo} & \textbf{Identificação} & \textbf{Ref.}\\
\midrule
\endfirsthead
\multicolumn{5}{l}{Continuação da página anterior} \\
\toprule

\textbf{Aula} & \textbf{Data} & \textbf{Conteúdo} & \textbf{Identificação} & \textbf{Ref.} \\

\midrule
\endhead
\midrule\multicolumn{5}{r}{Continua na página seguinte} \\
\endfoot
\endlastfoot
\rowcolor{green!25} & 23/08 & Não haverá aula: Recepção calouros &  & \\
\rowcolor{yellow!25} 01 & 25/08 & Objetivos, motivação da disciplina, conceitos básicos e ambiente de programação & Apresentação & \\
\rowcolor{gray!25} 02 & 30/08 & Variáveis, atribuições e operações aritméticas & Tópico 1 / Expositiva & \\
\rowcolor{yellow!25} 03 & 01/09 & Variáveis, atribuições e operações aritméticas & Tópico 1 / Interativa & \\
\rowcolor{gray!25} 04 & 06/09 & Comandos condicionais & Tópico 2 / Expositiva & \\
\rowcolor{yellow!25} 05 & 08/09 & Comandos condicionais & Tópico 2 / Interativa & \\
\rowcolor{gray!25} 06 & 13/09 & Revisão - Tópicos 1 e 2 &  & \\
\rowcolor{red!15} 07 & 15/09 & \textbf{Avaliação 01 - Tópicos 1 e 2} &  & \\
\rowcolor{gray!25} 08 & 20/09 & Comandos de repetição & Tópico 3 / Expositiva & \\
\rowcolor{yellow!25} 09 & 22/09 & Comandos de repetição & Tópico 3 / Interativa & \\
\rowcolor{gray!25} 10 & 27/09 & Funções & Tópico 4 / Expositiva & \\
\rowcolor{yellow!25} 11 & 29/09 & Funções & Tópico 4 / Interativa & \\
\rowcolor{gray!25} 12 & 04/10 & Revisão - Tópicos 3 e 4 &  & \\
\rowcolor{yellow!25} 13 & 06/10 & \textbf{Avaliação 02 - Tópicos 3 e 4} &  & \\
\bottomrule
\end{longtable}
\end{document}
